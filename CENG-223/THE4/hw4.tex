\documentclass[11pt]{article}
\usepackage[utf8]{inputenc}
\usepackage{float}
\usepackage{amsmath}
\usepackage{amssymb}

\usepackage[hmargin=3cm,vmargin=6.0cm]{geometry}
%\topmargin=0cm
\topmargin=-2cm
\addtolength{\textheight}{6.5cm}
\addtolength{\textwidth}{2.0cm}
%\setlength{\leftmargin}{-5cm}
\setlength{\oddsidemargin}{0.0cm}
\setlength{\evensidemargin}{0.0cm}

% symbol commands for the curious
\newcommand{\setZp}{\mathbb{Z}^+}
\newcommand{\setR}{\mathbb{R}}
\newcommand{\calT}{\mathcal{T}}

\begin{document}

\section*{Student Information } 
%Write your full name and id number between the colon and newline
%Put one empty space character after colon and before newline
Full Name : Berkay Bartuğ Çetin\\
Id Number : 2309839 \\

% Write your answers below the section tags
\section*{Answer 1}
We can choose 1 star, 2 habitable planets and 8 non-habitable planets in $\binom{10}{1}.\binom{20}{2}.\binom{80}{8}$ ways. \\
When we place 6 non-habitable planets at the center of 10 planets, there will be 2 empty places at the left and right sides of those 6 nonhabitable planets. \\
We can pick 1 habitable planet from 2 habitable planets in C(2,1) ways and we can put this planet to P(2,1) places. \\
This leads us to pick the last planet (C(1,1)) and place it to an empty place from other side (P(2,1)).\\
Other 8 nonhabitable may line up in P(8,8) ways. Then the total number of selection is; 2*2*2*8! \\
Then we need to think about the way that what if habitable planets gets the 1st-8th places and 3rd-10th places. \\
We can pick 1 habitable planet in C(2,1) ways and it will be pleaced to 1st place P(1,1), then other habitable planet (C(1,1)) will be placed to 8th place (P(1,1)). \\
And we ween to consider 3rd-10th case too. Hence total number of selections of those cases; 2*2*8!. \\
We need to sum those total number of selections up and then our final number of selecitons becomes 12*8! \\
Multiplying the selection and positioning of a galaxy with a given condition is the number of ways to shape it, hence it is, \\
$\binom{10}{1}.\binom{20}{2}.\binom{80}{8}.12.8!$ \\


\section*{Answer 2}
Homogeneous recurrence relation is $a_n = 2a_{n-1}+15a_{n-2}-36a_{n-3}+2^n$ \\
 Characteristic eq. is $ r^3 - 2r^2 - 15r + 36 = 0$\\
 $ (r-3)^2 . (r+4) = 0$\\
Characteristic roots are $ r = 3$ with multiplicity 2, and $r = -4$. Hence, homogeneous solution is,\\
 $a_n^{(h)} = \alpha_1 .3^n + \alpha_2 .n.3^n + \alpha_3 .(-4)^n $\\
 $F(n) = 2^n$ and 2 is not a root, particular solution is the form of $a_n^{(p)} = c.2^n$. Put $a_n^{(p)} = c.2^n$ into the recurrence relation and we obtain,\\
 $c.2^n = 2c.2^{(n-1)} + 15c.2^{(n-2)} -36c.2^{(n-3)} + 2^n$\\
 $=8c.2^{(n-3)} = 2^{(n-3)}.(2c.4 + 15c.2 - 36c + 8)$\\
 $c = 4/3$\\
 Hence, we get the particular solution as,\\
 $a_n^{(p)} = \dfrac{4}{3}.2^n = \dfrac{2^{(n+2)}}{3}$\\
 We can obtain the general solution from summing particular and homogeneous solutions.\\
 $a_n = \alpha_1 .3^n + \alpha_2 .n.3^n + \alpha_3 .(-4)^n +  \dfrac{2^{(n+2)}}{3}$\\


\newpage
\section*{Answer 3}
First we need to note that $a_1 = 5$ because there are 5 one digit odd numbers. \\
Recurrence relation can be derived by considering how a valid n-digit code may be obtained from $n-1$ digit codes. \\
An even number can be added to a $n-1$ digit code and we can do that in 5 different ways. \\
Hence there are $5.a_{n-1}$ ways to form a valid n-digit code. \\
Moreover, we can derive a valid n-digit code by adding an odd number to a n-1 digit non-valid code. \\
Since there exist $10^{n-1}$ codes of length n-1, and $a_{n-1}$ of them are valid, there are 5 odd numbers to add. \\
There is $5.(10^{n-1} - a_{n-1}) $ valid n-digit codes derived by adding an odd number to a non-valid code of length n-1. \\
All of the n digit valid codes have to be obtained from these ways. \\
Hence, there are $a_n = 5.a_{n-1} + 5.(10^{n-1} - a_{n-1})= 5.10^{n-1}=a_n$ valid codes. \\


\newpage
\section*{Answer 4}
Let $G\left(x\right)=\:\sum _{k=0}^{\infty }\:a_kx^k$ \\
$G\left(x\right)-a_0-a_1x-a_2x^2=\:\sum _{k=3}^{\infty }\:a_kx^k$ \\
$=\sum _{k=1}^{\infty }\:\left(3a_{k-1}-3a_{k-2}+a_{k-3}\right)x^k$ ($\:\left(a_k=3a_{k-1}-3a_{k-2}+a_{k-3}\right)$ when $k\ge1$) \\
$=3\:\sum \:_{k=3}^{\infty \:}\:\left(a_{k-1}x^k\right)-3\:\sum \:\:_{k=3}^{\infty \:\:}\:\left(a_{k-2}x^k\right)+\:\sum \:\:_{k=3}^{\infty \:\:}\:\left(a_{k-3}x^k\right)$ \\
$=3x\sum \:_{k=3}^{\infty \:}\:\left(a_{k-1}x^{k-1}\right)-3x^2\:\sum \:\:_{k=3}^{\infty \:\:}\:\left(a_{k-2}x^{k-2}\right)+\:x^3\sum \:\:_{k=3}^{\infty \:\:}\:\left(a_{k-3}x^{k-3}\right)$ \\
$(m=k-1,n=k-2, t=k-3)$ \\
$=3x\sum \:_{m=2}^{\infty \:}\:\left(a_mx^m\right)-3x^2\:\sum \:\:_{n=1}^{\infty \:\:}\:\left(a_nx^n\right)+\:x^3\sum \:\:_{t=0}^{\infty \:\:}\:\left(a_tx^t\right)$ \\
$=3x\left(G\left(x\right)-a_0-a_1x\right)-3x^2\left(G\left(x\right)-a_0\right)+x^3G\left(x\right)$  ($\sum _{n=0}^{\infty }\:x^n=\frac{1}{1-x}$) \\
We thus obtained the equation; \\
$G\left(x\right)-a_0-a_1x-a_2x^2=3x\left(G\left(x\right)-a_0-a_1x\right)-3x^2\left(G\left(x\right)-a_0\right)+x^3G\left(x\right)$ \\
$G\left(x\right)-1-3x-6x^2=3x\left(G\left(x\right)-1-3x\right)-3x^2\left(G\left(x\right)-1\right)+x^3G\left(x\right)$ $(a_0=1,\:a_1=3\:and\:a_2=6)$ \\
$G\left(x\right)-1-3x-6x^2=3xG\left(x\right)-3x-3x^2G\left(x\right)-9x^2+x^3G\left(x\right)$ (Distributive Propertey) \\
$Add\:-3xG\left(x\right)+3x^2G\left(x\right)-x^3G\left(x\right)\:and\:1+3x+6x^2\:to\:each\:side$ \\
$G\left(x\right)-3xG\left(x\right)+3x^2G\left(x\right)-x^3G\left(x\right)=1$ \\

Using $\:\sum \:_{k=1}^{\infty \:}\:C\left(2+k,k\right)x^k=\frac{1}{\left(1-x\right)^3}$ \\
$G\left(x\right)= \sum \:_{k=1}^{\infty \:}\:C\left(k+2,k\right)x^k$ \\
Then we note $a_k=C\left(2+k,k\right)$ \\
$a_k=\frac{\left(k+2\right)!}{k!2!}$ \\


\newpage
\section*{Answer 5}
\paragraph{a.}
R is reflexive ($a+b=b+a\:\Rightarrow \:\left(a,b\right)R\left(a,b\right)$), \\
R is symmetrix ($a+d=b+c\Rightarrow c+b=d+a\:\Rightarrow\left(c,d\right)R\:\left(a,b\right)$), \\
R is transitive ($a+d=b+c\:\:and\:c+f=d+e$) \\
$\Rightarrow a+d+c+f=b+c+d+e$,
$\Rightarrow a+f=b+e\Rightarrow\:\left(a,b\right)R\left(e,f\right)$. \\
Hence, R is a equivalence relation. 

\paragraph{b.}
According to description, $\left(\left(a,b\right),\left(1,2\right)\right)\in R$ if and only if $a+2=b+1\Rightarrow b=a+1$ \\
So, the equivalance class of (1,2) is; \\
$\left[\left(1,2\right)\right]_R=\:\left\{\left(a,a+1\right):a\in Z^+\right\}$ \\


\end{document}